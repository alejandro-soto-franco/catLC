\documentclass[a4paper, 11pt]{extarticle}
% \usepackage{fontspec}

% ==================================================

% document parameters
% \usepackage[spanish]{babel}
\usepackage[english]{babel}
\usepackage[margin = 1in]{geometry}
\usepackage{subcaption}
\usepackage{algorithm}
\usepackage{algpseudocode}
% ==================================================

% Packages for math
\usepackage{mathrsfs}
\usepackage{amsfonts}
\usepackage{amsmath}
\numberwithin{equation}{subsection}
\usepackage{amsthm}
\usepackage{amssymb}
\usepackage{physics}
\usepackage{dsfont}
\usepackage{esint}
\usepackage{tikz}
\usepackage{tikz-3dplot}
\usetikzlibrary{positioning, arrows.meta, backgrounds, fit, calc}

% ==================================================

% Packages for writing
\usepackage{enumerate}
\usepackage[shortlabels]{enumitem}
\usepackage{framed}
\usepackage[utf8]{inputenc}
\usepackage{csquotes}


% ==================================================

% Miscellaneous packages
\usepackage{float}
\usepackage{tabularx}
\usepackage{multicol}
\usepackage{subcaption}
\usepackage{caption}
\captionsetup{format = hang, margin = 10pt, font = small, labelfont = bf}

% Citation
\usepackage[numbers, square]{natbib}

% Hyperlinks setup
\usepackage{hyperref}
\usepackage{cleveref}
\usepackage{listings}
\usepackage{etoolbox}
\usepackage{xcolor}
\usepackage{quoting}
\usepackage{changepage} % for adjustwidth

\newenvironment{fancyquote}[1][]
{\begin{adjustwidth}{2em}{2em}%
		\itshape
		\ifstrempty{#1}{}{---\ #1\\[1ex]}}
	{\end{adjustwidth}\vspace{1em}}


\lstset{
    language=Python,
    basicstyle=\ttfamily\footnotesize,
    keywordstyle=\color{blue},
    commentstyle=\color{gray},
    stringstyle=\color{red},
    breaklines=true,
    numbers=left,
    numberstyle=\tiny,
    stepnumber=1,
    numbersep=5pt,
    frame=single,
    captionpos=b,
    tabsize=4,
    showstringspaces=false,
}

\definecolor{links}{rgb}{0.36,0.54,0.66}
\hypersetup{
    colorlinks = false,
    linkcolor = red,
    urlcolor = blue,
    citecolor = blue,
    filecolor = blue,
    pdfauthor = {Author},
    pdftitle = {Title},
    pdfsubject = {subject},
    pdfborder = {0 0 1},
    linkbordercolor = {1 0 0},
    citebordercolor = {1 0 0},
    urlbordercolor  = {1 0 0},
    pdfkeywords = {one, two},
    pdfproducer = {LaTeX},
    pdfcreator = {pdfLaTeX},
   }

\theoremstyle{plain}
\newtheorem{theorem}{Theorem}[section]
\newtheorem{lemma}[theorem]{Lemma}
\newtheorem{proposition}[theorem]{Proposition}
\newtheorem{corollary}{Corollary}

\theoremstyle{definition}
\newtheorem{definition}{Definition}[section]
\newtheorem{axiom}{Axiom}[section]
\newtheorem{conjecture}{Conjecture}[section]
%\newtheorem{example}{Example}[section]

\theoremstyle{remark}
\newtheorem{remark}{Remark}
\newtheorem*{note}{Note}

\usepackage{mdframed}
% Define colors
\definecolor{formalshade}{rgb}{0.95,0.95,1} % Light blue background
\definecolor{darkblue}{rgb}{0.0, 0.0, 0.55} % Border color
\newmdenv[
  leftmargin=4pt,
  rightmargin=4pt,
  backgroundcolor=formalshade,
  linecolor=darkblue,
  linewidth=2pt,
  roundcorner=4pt,
  innerleftmargin=6pt,
  innerrightmargin=6pt,
  innertopmargin=6pt,
  innerbottommargin=6pt,
  skipabove=6pt,
  skipbelow=6pt,
  nobreak=true, % Allows spanning pages
]{formal}

\usepackage{booktabs}  % For \toprule, \midrule, \bottomrule
\usepackage{array}     % For improved table formatting

\newcommand{\notimplies}{\;\not\!\!\!\implies}

%%%%%%%%%%%%%%%%%%%%%%%%% HEADER AND FOOTER
\usepackage{fancyhdr}
\pagestyle{fancy}
\fancypagestyle{plain}
\fancyhf{}
\lhead{ASF}
\rhead{DBL SP25}

% --- Basic commands ---
%   Euler's constant
\newcommand{\eu}{\mathrm{e}}

%   Imaginary unit
\newcommand{\im}{\mathrm{i}}

%   Sexagesimal degree symbol
\newcommand{\grado}{\,^{\circ}}

% --- Comandos para álgebra lineal ---
% Matrix transpose
\newcommand{\transpose}[1]{{#1}^{\mathsf{T}}}

%%% Comandos para cálculo
%   Definite integral from -\infty to +\infty
\newcommand{\Int}{\int\limits_{-\infty}^{\infty}}

%   Indefinite integral
\newcommand{\rint}[2]{\int{#1}\dd{#2}}

%  Definite integral
\newcommand{\Rint}[4]{\int\limits_{#1}^{#2}{#3}\dd{#4}}

%   Dot product symbol (use the command \bigcdot)
\makeatletter
\newcommand*\bigcdot{\mathpalette\bigcdot@{.5}}
\newcommand*\bigcdot@[2]{\mathbin{\vcenter{\hbox{\scalebox{#2}{$\m@th#1\bullet$}}}}}
\makeatother

%   Hamiltonian
\newcommand{\Ham}{\hat{\mathcal{H}}}

%   Trace
\renewcommand{\Tr}{\mathrm{Tr}}

% Christoffel symbol of the second kind
\newcommand{\christoffelsecond}[4]{\dfrac{1}{2}g^{#3 #4}(\partial_{#1} g_{#2 #4} + \partial_{#2} g_{#1 #4} - \partial_{#4} g_{#1 #2})}

% Riemann curvature tensor
\newcommand{\riemanncurvature}[5]{\partial_{#3} \Gamma_{#4 #2}^{#1} - \partial_{#4} \Gamma_{#3 #2}^{#1} + \Gamma_{#3 #5}^{#1} \Gamma_{#4 #2}^{#5} - \Gamma_{#4 #5}^{#1} \Gamma_{#3 #2}^{#5}}

% Covariant Riemann curvature tensor
\newcommand{\covariantriemanncurvature}[5]{g_{#1 #5} R^{#5}{}_{#2 #3 #4}}

% Ricci tensor
\newcommand{\riccitensor}[5]{g_{#1 #5} R^{#5}{}_{#2 #3 #4}}

% \renewcommand{\familydefault}{\sfdefault}

% Boldface unit vector with a hat
\newcommand{\uv}[1]{\mathbf{\hat{#1}}}

\newcommand{\ep}{\varepsilon}

\DeclareMathOperator{\sgn}{sgn}

\usepackage{titlesec}
\usepackage[many]{tcolorbox}

% Adjust spacing after the chapter title
\titlespacing*{\chapter}{0cm}{-2.0cm}{0.50cm}
\titlespacing*{\section}{0cm}{0.50cm}{0.25cm}

% Indent 
\setlength{\parindent}{0pt}
\setlength{\parskip}{1ex}

% --- Theorems, lemma, corollary, postulate, definition ---
% \numberwithin{equation}{section}
\tcbuselibrary{skins, breakable,theorems}
\newtcbtheorem[]{problem}{Problem}%
    {enhanced,
    colback = black!5, %white,
    colbacktitle = black!5,
    coltitle = black,
    boxrule = 0pt,
    frame hidden,
    borderline west = {0.5mm}{0.0mm}{black},
    fonttitle = \bfseries\sffamily,
    breakable,
    before skip = 3ex,
    after skip = 3ex
}{problem}
\makeatletter
\let\example\relax
\makeatother
% Forcefully undefine \example (this is sometimes needed if it’s already defined)
\makeatletter
\let\example\relax
\makeatother

% Now define the 'example' environment using tcolorbox theorem style:
\newtcbtheorem[]{example}{Example}%
  {enhanced,
   colback = black!5, % background color
   colbacktitle = black!5, % title background color
   coltitle = black, % title text color
   boxrule = 0pt, % no box border
   frame hidden,
   borderline west = {0.5mm}{0.0mm}{black},
   fonttitle = \bfseries\sffamily,
   breakable,
   before skip = 3ex,
   after skip = 3ex
  }{example}
\newcommand{\vphi}{\varphi}

% --- You can define your own color box. Just copy the previous \newtcbtheorm definition and use the colors of yout liking and the title you want to use.

\newcommand{\bb}{\mathbb}
\usepackage[bottom]{footmisc}
%% Commands for PSET6

\begin{document}
	
	\begin{center}
		\begin{huge}
			{\textbf{\texttt{catLC}: Multiscale Categorical Field Theory for Liquid Crystals}}\\[2ex]
		\end{huge}
		
		\begin{Large}
			by Alejandro Soto Franco\footnote{\texttt{asoto12@jhu.edu, sotofranco.math@gmail.com}}, last updated \today
		\end{Large}
	\end{center}
	
	\thispagestyle{empty}
	
	\tableofcontents
	
	\newpage
	
	\section{Introduction}
	
	Information underlies every physical model. Whether encoded in molecular configurations, order parameter fields, or large-scale defect networks, nature computes its behavior through structured data. As Deutsch explores across various constants for universe iterations in his \emph{The Fabric of Reality} \cite{deutsch1997fabric}, information is encoded in regular patterns that can be physically realized through the identification of invariances up to a particular resolution scale. In liquid crystal theory, this viewpoint connects phenomena across scales, from microscopic interactions to macroscopic defect networks, as detailed in \cite{degennes1993physics, cardy1996scaling, ma1976modern}.
	
	A powerful tool for understanding multiscale phenomena is the \textbf{renormalization group (RG)}. RG flow describes how system parameters evolve as one “zooms out” and coarse-grains microscopic details. In the context of liquid crystals, local molecular interactions are aggregated into effective mesoscopic order parameters and further into macroscopic defect configurations. This coarse-graining procedure not only reveals fixed points corresponding to phase transitions but also uncovers the propagation of local fluctuations across scales.
	
	Inspired by the structural insights of category theory \cite{maclane1971categories, spivak2014category} and by developments in topological quantum field theory \cite{atiyah1988topological, segal2004definition, freed1995chern}, we propose a \emph{multiscale categorical field theory for liquid crystals}. In this framework:
	\begin{itemize}
		\item \textbf{Objects} represent complete informational states at various scales—from microscopic molecular configurations through mesoscopic \(Q\)-tensor fields to macroscopic defect networks \cite{degennes1993physics}.
		\item \textbf{Morphisms} encode the dynamical or topological transitions between these states, including the RG transformations that systematically integrate out degrees of freedom \cite{cardy1996scaling, ma1976modern}.
		\item \textbf{Functors} serve to relate and translate information between different levels of description, much like RG maps relate theories at distinct scales \cite{baez2011rosetta, abramsky2004categorical}.
	\end{itemize}
	
	Moreover, our approach establishes links with higher gauge theories \cite{baez2007higher} and categorical semantics in quantum protocols \cite{selinger2007dagger, abramsky2004categorical}, emphasizing the universality of compositional principles across physics.
	
	\section{Categorical Renormalization Group Flows}
	
	Consider a microscopic state \(A\) of a liquid crystal system, characterized by detailed molecular configurations. The RG flow defines a map 
	\[
	R: A \to B,
	\]
	where \(B\) represents a mesoscopic state described by an effective \(Q\)-tensor field. Repeated application of the RG transformation leads to a sequence of states
	\[
	A \xrightarrow{R_1} B \xrightarrow{R_2} C \xrightarrow{R_3} \cdots,
	\]
	with each \(R_i\) acting as a morphism in the category \(\mathcal{C}\) of informational states. The composition
	\[
	R_{n} \circ \cdots \circ R_2 \circ R_1
	\]
	then represents the full RG flow from the microscopic to the macroscopic scale. By the associativity of morphism composition \cite{maclane1971categories}, this overall transformation is well-defined regardless of the grouping of intermediate steps.
	
	\subsection{Functoriality and Structure Preservation}
	
	Within our framework, the RG transformation is treated as a functor:
	\[
	\mathcal{R}: \mathcal{C}_{\text{micro}} \to \mathcal{C}_{\text{meso}},
	\]
	mapping microscopic theories to effective mesoscopic descriptions. This functor satisfies:
	\begin{enumerate}[label=(\alph*)]
		\item For every object \(A \in \mathcal{C}_{\text{micro}}\), the image \(\mathcal{R}(A)\) is an object in \(\mathcal{C}_{\text{meso}}\) that encapsulates coarse-grained information \cite{spivak2014category}.
		\item For every morphism \(f: A \to B\) in \(\mathcal{C}_{\text{micro}}\), the functor assigns a morphism \(\mathcal{R}(f): \mathcal{R}(A) \to \mathcal{R}(B)\) such that
		\[
		\mathcal{R}(g \circ f) = \mathcal{R}(g) \circ \mathcal{R}(f),
		\]
		thereby preserving the compositional structure \cite{atiyah1988topological, segal2004definition}.
	\end{enumerate}
	
	This functorial view parallels constructions in topological quantum field theory \cite{freed1995chern} and higher gauge theory \cite{baez2007higher}, ensuring that the underlying informational structure is maintained across scales.
	
	\subsection{Computational Implementation of the RG Flow}
	
	The RG transformation can be decomposed into two primary operations:
	\begin{enumerate}[label=(\arabic*)]
		\item \textbf{Local Averaging (Map Operation):} For each local region \(R\) in the microscopic state, compute an order parameter \(q_R\) that summarizes the local configuration. This step corresponds to integrating out high-frequency modes \cite{degennes1993physics}.
		\item \textbf{Aggregation (Reduce Operation):} Combine the locally computed order parameters \(q_R\) to form the effective mesoscopic state \(B\) by aggregating local contributions into a global structure \cite{hudak1989conception}.
	\end{enumerate}
	
	Mathematically, if the microscopic state is partitioned into regions \( \{ R_1, R_2, \dots, R_N \} \) with associated order parameters
	\[
	q_{R_i} = \texttt{compute\_order}(R_i),
	\]
	the mesoscopic state is constructed as
	\[
	B = \texttt{aggregate}(\{q_{R_1}, q_{R_2}, \dots, q_{R_N}\}).
	\]
	
	An iterative RG flow is then represented as:
	\[
	A^{(0)} \xrightarrow{R^{(1)}} A^{(1)} \xrightarrow{R^{(2)}} A^{(2)} \xrightarrow{R^{(3)}} \cdots,
	\]
	with convergence toward a fixed point \(A^{(\infty)}\) satisfying
	\[
	R(A^{(\infty)}) \cong A^{(\infty)}.
	\]
	
	\subsection{Rust Implementation Example}
	
	The following Rust code illustrates a preliminary implementation of a single RG transformation step. The implementation reflects a functional programming style reminiscent of Haskell \cite{hudak1989conception}:
	
	\begin{lstlisting}[caption={Rust Implementation of a Single RG Transformation Step}]
		#[derive(Debug)]
		struct Region {
			// Microscopic configuration in a local region.
			data: Vec<f64>,
		}
		
		#[derive(Debug)]
		struct MicroscopicState {
			// Collection of local regions.
			regions: Vec<Region>,
		}
		
		#[derive(Debug)]
		struct OrderParameter {
			// Local order parameter computed for a region.
			value: f64,
		}
		
		#[derive(Debug)]
		struct MesoscopicState {
			// Aggregated mesoscopic state.
			order_parameters: Vec<OrderParameter>,
		}
		
		/// Computes the local order parameter for a given region.
		fn compute_order(region: &Region) -> OrderParameter {
			let sum: f64 = region.data.iter().sum();
			let avg = sum / (region.data.len() as f64);
			OrderParameter { value: avg }
		}
		
		/// Aggregates local order parameters into a mesoscopic state.
		fn aggregate(order_parameters: Vec<OrderParameter>) -> MesoscopicState {
			MesoscopicState { order_parameters }
		}
		
		/// Performs a single RG transformation step.
		fn rg_step(microscopic: &MicroscopicState) -> MesoscopicState {
			let order_parameters: Vec<OrderParameter> = microscopic.regions
			.iter()
			.map(|region| compute_order(region)) /// where a lot of the computational work of our lab and its collaborators has been focused
			.collect();
			aggregate(order_parameters)
		}
		
		fn main() {
			let region1 = Region { data: vec![1.0, 2.0, 3.0] };
			let region2 = Region { data: vec![4.0, 5.0, 6.0] };
			let region3 = Region { data: vec![7.0, 8.0, 9.0] };
			let microscopic = MicroscopicState {
				regions: vec![region1, region2, region3],
			};
			let mesoscopic = rg_step(&microscopic);
			println!("Mesoscopic state: {:?}", mesoscopic);
		}
	\end{lstlisting}
	
	\section{Methods}
	
	\subsection{Categorical Foundations}
	
	A \emph{category} \(\mathcal{C}\) consists of:
	\begin{enumerate}[label=(\roman*)]
		\item A collection \(\mathrm{Ob}(\mathcal{C})\) of \textbf{objects}, each representing an informational state (e.g., a detailed \(Q\)-tensor field, director configuration, or defect network) \cite{maclane1971categories, spivak2014category}.
		\item For any two objects \(A,B \in \mathrm{Ob}(\mathcal{C})\), a set \(\mathrm{Hom}_{\mathcal{C}}(A,B)\) of \textbf{morphisms} that represent transitions or processes from \(A\) to \(B\) \cite{maclane1971categories}.
		\item A composition law: For any morphisms \(f \in \mathrm{Hom}_{\mathcal{C}}(A,B)\) and \(g \in \mathrm{Hom}_{\mathcal{C}}(B,C)\), the composite morphism \(g\circ f \in \mathrm{Hom}_{\mathcal{C}}(A,C)\) satisfies
		\[
		h\circ (g\circ f) = (h\circ g)\circ f,
		\]
		for all choices of \(f\), \(g\), and \(h\) \cite{maclane1971categories}.
		\item For each object \(A\), an identity morphism \(\mathrm{id}_A\) satisfying
		\[
		f\circ \mathrm{id}_A = f \quad \text{and} \quad \mathrm{id}_B\circ f = f,
		\]
		for every \(f \in \mathrm{Hom}_{\mathcal{C}}(A,B)\).
	\end{enumerate}
	
	This structure is analogous to those found in topological and symplectic categories \cite{weinstein1981symplectic, guillemin1984symplectic}.
	
	\subsection{Informational Axiom and Conjecture}
	
	\begin{axiom}[Informational Conjecture]
		Every object in \(\mathcal{C}\) is a repository of information. In liquid crystal physics, an object may encode microscopic molecular configurations, mesoscopic \(Q\)-tensor fields, or macroscopic defect networks \cite{degennes1993physics, spivak2014category}.
	\end{axiom}
	
	\begin{remark}
		Morphisms in \(\mathcal{C}\) represent the dynamical or topological transitions between informational states. Their composition mirrors the pure function composition in functional programming \cite{hudak1989conception} and extends naturally to categorical semantics in quantum protocols \cite{abramsky2004categorical, selinger2007dagger}.
	\end{remark}
	
	\begin{theorem}[Associativity of Compositional Transformations]
		For any morphisms \(f: A \to B\), \(g: B \to C\), and \(h: C \to D\) in \(\mathcal{C}\), we have
		\[
		h\circ (g\circ f) = (h\circ g)\circ f.
		\]
	\end{theorem}
	
	\begin{proof}
		This follows directly from the definition of a category \cite{maclane1971categories}.
	\end{proof}
	
	\subsection{Extensions to Evolving Geometries and Stochastic Dynamics}
	
	Beyond standard RG flows, our framework accommodates evolving geometries and stochastic dynamics. Here, the microscopic theories are defined on a high-dimensional manifold \(M\) whose geometry evolves via curvature-driven flows (e.g., mean curvature flow or Ricci flow) \cite{hamilton1982three, grayson1987shortening}. 
	
	To capture stochastic evolution, we introduce Markov chain dynamics governed by a probability kernel \(P\) \cite{levin2009markov}. In this enriched setting, the RG flow functor
	\[
	\mathcal{R}: \mathcal{C}_{\text{micro}}^G \to \mathcal{C}_{\text{meso}}^G,
	\]
	acts as a Markov operator with the following properties:
	\begin{enumerate}[label=(\alph*)]
		\item Each microscopic object \((M, \phi)\) is mapped to an effective mesoscopic object \(\mathcal{R}(M, \phi) = (\widetilde{M}, \widetilde{\phi})\) through geometric smoothing and statistical averaging \cite{e2003multiscale}.
		\item For every morphism \(f: (M, \phi) \to (M', \phi')\) with transition probability \(P\big((M, \phi), (M', \phi')\big)\), the functor assigns a corresponding morphism \(\mathcal{R}(f)\) that preserves the probabilistic structure.
	\end{enumerate}
	
	This approach aligns with methods in topological quantum field theory \cite{atiyah1988topological, guillemin1984symplectic} and provides a robust framework for exploring universality in complex systems.
	
	\section{Conclusion}
	
	We have developed a multiscale categorical field theory for liquid crystals that integrates renormalization group flows with the compositional structure of category theory. By interpreting RG transformations as functors, we preserve the informational content across scales and link microscopic details with macroscopic phenomena. Our framework draws on classical foundations \cite{maclane1971categories, degennes1993physics} and modern perspectives from topological and higher gauge theories [\cite{atiyah1988topological, segal2004definition, baez2007higher, freed1995chern}]. Furthermore, connections with functional programming paradigms [\cite{hudak1989conception}] and categorical semantics in quantum protocols [\cite{abramsky2004categorical, selinger2007dagger}] underscore the broad applicability of these ideas.
	
	This work opens new avenues for research into the interplay between geometry, topology, and stochastic dynamics in liquid crystals, and more broadly, in complex physical systems.
	
	\newpage
	\bibliographystyle{apalike}
	\bibliography{references}
	
\end{document}
